\newpage
\thispagestyle{empty}

\chapter*{Samenvatting}
\addcontentsline{toc}{part}{Samenvatting}
\setheader{Samenvatting}
\label{Samenvatting}


Vaste stoffen bestaan uit elektronen en ionen. De meeste vaste stoffen kunnen effectief worden gemodelleerd met een theorie waarin de beweging van de elektronen allemaal als onafhankelijk van elkaar kan worden beschouwd. Dit proefschrift houdt zich bezig met enkele gevallen waarin deze beschrijving onvermijdelijk faalt, en meerdere-deeltjes-interacties de lage-energiefysica domineren. Dergelijke systemen zijn veel moeilijker te bestuderen. Soms kan vooruitgang worden geboekt met behulp van geavanceerde numerieke technieken, of door creatief geconstrueerde, precies oplosbare speelgoedmodellen.
\par
Dit laatste is de benadering van het Sachdev-Ye-Kitaev (SYK)-model, dat een kwantumdot beschrijft met een groot aantal fermionen erop, die allemaal met elkaar interacteren door middel van een gerandomiseerde interactie met vier fermionen. De eenvoud van het model verhult een overvloed aan exotische verschijnselen die voortkomen uit een emergente conforme symmetrie in het infrarood, met eclectische toepassingen van vreemde metalen tot zwarte gaten. Door zogenaamde holografische dualiteit kan de lage-energietheorie van het model, die voortkomt uit het verbreken van de bovengenoemde conforme symmetrie, in kaart worden gebracht via verstoringen van een bijna AdS${}_2$ ruimtetijd. 
\par
Eén van die verschijnselen staat bekend als maximale chaos. Het SYK-model verzadigt een grens aan de snelheid waarmee in een kwantumsysteem verstoringen kunnen dissiperen in zijn vrijheidsgraden. Dit kenmerk blijft bestaan, zelfs als men bepaalde gewijzigde versies van het SYK-model construeert, zoals het bipartiete SYK-model. De mate van chaos verandert niet, zelfs niet als de schaaldimensies van de samenstellende fermionen worden veranderd, zoals beschreven in hoofdstuk ~\ref {ch:LyapbSYK}. De eerste subleidende correctie van de chaos exponent blijft ook onafhankelijk te zijn van de schaaldimensie.
\par
Wanneer twee identieke kopieën van het SYK-model aan elkaar worden gekoppeld door een tunnelinteractie, bij lage temperatuur beschrijft het holografische duaal de ruimtetijd van een wormgat. Een op een dergelijke manier gemaakt Josephson-contact kan worden bestudeerd met behulp van de SYK-versie van het elektron-fonon-gekoppelde systeem, dat bekend staat als het Yukawa SYK-model. Supergeleiding in het Yukawa SYK-model vindt plaats bij lage temperaturen als het systeem tijdinversie niet schendt. In hoofdstuk~\ref{chap:JosephsonWormhole} wordt aangetoond dat de niet-supergeleidende toestand de correcte niet-analytische afhankelijkheid van de "gap" vertoont van de sterkte van de tunnelinteractie, kenmerkend voor de wormgattoestand. Er wordt aangetoond dat de supergeleidende toestand effectief kan worden beschreven door een twee-vloeistoffenmodel, waarbij de Cooperparen onafhankelijk aan beide zijden leven, terwijl excitaties van ongebonden elektronen nog steeds de wormgattoestand vormen.
\par
Als tegenpool van de hyperbolische geometrie van de AdS${}_2$ ruimte, hebben we in het tweede deel van dit proefschrift een systeem beschouwd waarvan het niet-interagerende fermi-oppervlak is samengesteld uit een familie van hyperbolen. In gedraaid dubbellaags grafeen voor kleine draaihoeken is dit te zien in de vorm van van Hove-singulariteiten, die precies onder de magische hoek een hogere orde krijgen. In hoofdstuk~\ref{ch:KondoTBG} wordt aangetoond dat een kenmerk voor het waarnemen van het effect ervan in een realistische fysieke omgeving de reactie is op een magnetische onzuiverheid. De onzuiverheidsentropie vertoont duidelijke kenmerken nabij de magische hoek, en toont stromingen naar de verschillende soorten vaste punten die beschikbaar zijn in de faseruimte van het systeem. Totdat de magische hoek is bereikt, werkt het bestaan van de Dirac-kegel bij de laagste energieën als een afstootmiddel voor Kondo-screening, wat zich manifesteert als een entropie van $\log 2$ per onzuiverheid. Bij de magische hoek domineert de verschijning van de hogere orde Van Hove-singulariteit de bandstructuur, en dit leidt tot een versterkt Kondo-effect tot de laagst mogelijke energieschalen.

\newpage
\thispagestyle{empty}