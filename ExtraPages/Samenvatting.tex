\newpage
\thispagestyle{empty}

\chapter*{Samenvatting}
\addcontentsline{toc}{part}{Samenvatting}
\setheader{Samenvatting}
\label{Samenvatting}

Vooruitgang op het gebied van computerhardware wordt doorgaans opgevolgd door toepassingen die proberen de nieuwe hardware maximaal te benutten. Deze vooruitgangen vinden ook hun weg naar de wetenschap, waar ze ons helpen om de bestaande grenzen van wat mogelijk is te verleggen. 

Aan het begin van dit proefschrift, in hoofdstuk \ref{ch:therm}, contrasteren we de huidige consensus over thermalisatie in gesloten kwantumsystemen, de ``eigentoestand thermalisatie hypothese'' (ETH), tegenover een recent ontdekte ``operator thermalisatie hypothese'' (OTH), doormiddel van het bestuderen van thermalisatiedynamiek in gesloten unitaire kwantumsystemen. We hebben aangetoond hoe de twee hypotheses verschillend zijn, en toch in sommige opzichten vergelijkbaar. No-go-voorwaarden die worden opgelegd door de OTH zijn een kenmerk van de integreerbaarheid van de theorie, en slechts een kleine beweging van deze voorwaarden vandaan zorgt ervoor dat de matrixelementen hun ETH-vorm benaderen. Voor het oplossen van deze grote ``eigen-problems'' was rekenkracht nodig van het lokale computerraster van de Universiteit Leiden.

Een van de meest alomtegenwoordige numerieke methoden in de wetenschap is de Monte Carlo-simulatie, oorspronkelijk ontwikkeld door Stanisław Ulam toen hij werkte aan kernwapens in het Los Alamos National Laboratory. De hoofdgedachte van Monte Carlo simulatie is het willekeurig nemen van steekproeven van de waarden van een integrand om bij benadering de waarde van een integraal te berekenen. Hiermee heeft Monte Carlo-simulatie heeft een revolutie teweeggebracht in de wetenschap en het hedendaagse computing. In hoofdstuk \ref{ch:higgs} bestuderen we de zogenaamde ``rooster-ijktheorie''. Voortbouwend op de fundamenten die Wegner heeft gelegd bij zijn realisatie van de pure $Z_2$-ijktheorie, en een uitbreiding van het werk van Fradkin en Shenker, construeren we $Z_2$-ijktheorie gekoppeld aan de verschillende materievelden. Ondanks het feit dat ze niet interageren, leiden deze materievelden tot interessante verschijnselen wanneer ze worden doorgemeten met hetzelfde ijkveld. Er is namelijk een nieuwe ``registry'' -orde in de Higgs-fase ontstaan, wat betekent dat lokaal verschillende kopieën van materievelden hun vectoren parallel en anti-parallel uitlijnen, zelfs in de aanwezigheid van continue $O(2)$-symmetrie van materie velden. \newpage Door Monte Carlo-simulaties uit te voeren voor grotere 3D-roosters hebben we een aantal interessante kenmerken ontdekt van faseovergangen die voorheen overschaduwd werden door de effecten van eindige grootte.

De afgelopen jaren zijn we getuige geweest van een overweldigende ontwikkeling van machine learning-technieken, geïnspireerd door de nieuwe generaties grafische kaarten. De industrie leidde voornamelijk het onderzoek naar nieuwe toepassingen; toch hebben deze nieuwe en enerverende toepassingen hun weg gevonden naar de wetenschap, vooral de natuurkunde. Een methode, ``genaamd'' neurale kwantumtoestanden (NQS), betreft het gebruik van een neuraal netwerk om toestanden van zeer verstrengelde kwantum toestanden te representeren. Een bijzondere architectuur van neurale netwerken, genaamd ``Restricted Boltzmann-machines'' (RBM), is zeer geschikt voor de taak, gedeeltelijk omdat het niet-lokale correlatie dankzij het ontwerp omvat. In hoofdstuk \ref{ch:EE} onderzoeken we verstrengelingsentropie en de schaling ervan voor dezelfde ijktheorieën als in hoofdstuk \ref{ch:higgs}, nu uitgedrukt als een 2D-kwantumtheorie in één dimensie lager. We stellen vast dat de verwachte lineaire relatie tussen het totale aantal materievelden en verstrengelingsentropie niet aanwezig is.

De manier waarop de natuurkunde de toepassingen van machine learning heeft omarmd, kan ook andersom worden gebruikt, om enkele van de successen te rechtvaardigen en deze verder te verbeteren. In navolging van deze mantra passen we in hoofdstuk \ref{ch:edgeofchaos} de inzichten uit de statistische fysica toe om de computationele mechanica van diepe neurale netwerken te bestuderen. We onderzoeken de wijze waarop de initiële parameterverdeling voor de gewichten en biases kan leiden naar twee verschillende faseregimes van het netwerk, en hoe het kiezen van het optimale punt binnen dit fasediagram de uiteindelijke nauwkeurigheid van het netwerk bepaalt, onder de voorwaarde van een gelijke trainingstijd. We stellen vast dat het initialiseren van gewichten en biases volgens de lijn van faseovergang een noodzakelijke maar geen voldoende voorwaarde is voor optimale trainbaarheid.




\newpage
\thispagestyle{empty}