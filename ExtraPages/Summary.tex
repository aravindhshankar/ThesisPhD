\newpage
\thispagestyle{empty}

\chapter*{Summary}
\addcontentsline{toc}{part}{Summary}
\setheader{Summary}
\label{Summary}

Solids are composed of electrons and ions. Many solids can be effectively modeled by considering a free electron theory. This thesis is concerned with some cases where that description inevitably fails, and many body interactions dominate the low energy physics. Such systems are far more difficult to study, and sometimes progress can be made either with sophisticated numerical tools, or by creatively constructed exactly solvable toy models.
\par
The latter is the approach taken by the Sachdev-Ye-Kitaev (SYK) model, which describes a quantum dot with a large number of fermions on it, all interacting with each other by means of a randomized four-fermion interaction. The simplicity of the model belies a plethora of exotic phenomena that arise from an emergent conformal symmetry in the infrared, with eclectic applications from strange metals to black holes. Through the holographic duality, the low energy theory of the model arising from the breaking of the aforementioned conformal symmetry can be mapped to perturbations of a nearly AdS${}_2$ spacetime. 
\par
One such phenomenon is known as maximal chaos. The SYK model saturates a bound on the rate at which a quantum system can dissipate perturbations into its degrees of freedom. This feature persists even when one constructs certain modified versions of the SYK model, such as the bipartite SYK model, and the rate of chaos doesn't change even when the scaling dimensions of the constituent fermions are changed as is described in chapter~\ref{ch:LyapbSYK}. The first subleading to conformal correction to the chaos exponent is also found to be independent of the scaling dimension. 
\par
When two identical copies of the SYK model are coupled together by a tunneling interaction, its holographic dual describes a wormhole spacetime at low temperature. A Josephson contact made in such a way can be studied using the SYK version of the electron-phonon coupled system, which is known as the Yukawa SYK model. Superconductivity in the Yukawa SYK model occurs upon on the restoration of time reversal symmetry at low temperatures. In chapter~\ref{chap:JosephsonWormhole}, it was shown that the non-superconducting state shows the correct non-analytic dependence of the gap on the strength of the tunneling interaction, characteristic of the wormhole state. The superconducting state was shown to be effectively described by a two-fluid model, where the cooper pairs live independently on the two sides, while single electron excitations still form the wormhole state. 
\par
Departing from the hyperbolic geometry of the AdS${}_2$ space, we considered a system whose non-interacting fermi surface was composed of a family of hyperbolae in the second part of this thesis. In Twisted bilayer graphene for small twists, this is seen in the form of van Hove singularities, which turn higher order at precisely the magic angle. In chapter~\ref{ch:KondoTBG}, a probe for sensing its effect in a realistic physical setting was shown to be the in response of a magnetic impurity. The impurity entropy shows distinct signatures near the magic angle, showing flows to the different kinds of fixed points available in the phase space of the system. Until the magic angle is reached, the existence of the Dirac cone at the lowest energies acts as a deterrent for Kondo screening, which manifests itself as a $\log 2$ entropy per impurity. At the magic angle however, the appearance of the higher order van Hove singularity dominates the band structure, and it leads to an enhanced Kondo effect until the lowest energy scales possible.


%perhaps turn the following off if the summary lasts two pages
\newpage
\thispagestyle{empty}