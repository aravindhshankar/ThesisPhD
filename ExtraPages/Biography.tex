\newpage
\thispagestyle{empty}

\chapter*{Curriculum Vitae} %important is to use the * after chapter. This way there is not an extra chapter nummer added to these sections
\addcontentsline{toc}{part}{Curriculum Vitae}
\setheader{Curriculum Vitae}
\label{Curriculum Vitae}

I was born on September 26${}^{\text{th}}$, 1996 in Chennai, a big city in the south of India. After finishing high school, I went to West Bengal for a five-year integrated Bachelor-Master program in Physics at the Indian Institute of Technology, Kharagpur.
\par
During this period, I took courses in several areas of physics and engineering, including high energy physics and gravitational physics, with my favorite being condensed matter and statistical physics. I was also fortunate to get an early start at doing my own research by being selected for several internships throughout my bachelor and master studies. With Prof. Pinaki Sengupta at NTU Singapore, I worked on writing my first quantum Monte Carlo code to investigate the Heisenberg XXZ model on ladder-like systems. I then worked briefly with Prof. Barbara Terhal and Prof. Fabian Hassler on DMRG simulations at RTWH aachen, and then on numerical and analytical studies on the Anderson model on the Bethe lattice with Prof. Antonello Scardicchio at the ICTP, Trieste, looking for signatures of anomalous diffusion. I concluded my master studies with a thesis titled ''Disordered Superconductors'', working with Prof. Sudhansu Sekhar Mandal. 
\par
I started my PhD at Leiden university in the Quantum matter group of Prof. Jan Zaanen and Prof. Koenraad Schalm in August 2019. After a short stint there, I moved to the group of Prof. Vadim Cheianov in April 2021. It was also at this time that I started working very closely with Prof. Lars Fritz and his group at Utrecht University. 
\par
As the early years of my PhD programme were marred by the coronavirus pandemic, I attended several condensed matter schools online, including editions of the Princeton PSSC summer school and the Maglab winter school. After the pandemic ended, I was able to present my work at several schools and conferences, including poster presentations in Trieste, Italy, at the ICTP conference "From quantum criticality to flat bands", at the school "Emergence of quantum phases in novel materials", at CSIC, Madrid, Spain, at the Veldhoven Physics conference in the Netherlands, and an online talk at the APS March meeting in 2023, among others. I also had the opportunity to TA several undergraduate and graduate courses, an in particular to write a set of extensive lecture notes on Information Geometry with Prof. Subodh Patil. 
\par
After the defense of my PhD, I will move on to begin a Postdoctoral position at the International Center for Theoretical Physics(ICTP) in Trieste, Italy. 

\newpage
\thispagestyle{empty}