\newpage
\thispagestyle{empty}

\chapter*{Curriculum Vitae} %important is to use the * after chapter. This way there is not an extra chapter nummer added to these sections
\addcontentsline{toc}{part}{Curriculum Vitae}
\setheader{Curriculum Vitae}
\label{Curriculum Vitae}

I was born on September 26th, 1996 in Chennai, a big city in the south of India. After finishing high school, I went to West Bengal for a five-year integrated Bachelor-Master program in Physics at the Indian Indian Institute of Technology, Kharagpur.
\par
High school was the place where I discovered that I wanted to study physics later in life. Besides the regular high school curriculum, I attended the physics seminar at Petnica science center. I was quickly introduced to the world of computer simulations and developed a particle-in-cell (PIC) code for simulating a plasma wave accelerator. After finishing high school, I moved to Belgrade, where I obtained my bachelor's degree. In the first year, my publication ``New Approaches for Boosting to Uniformity'' was awarded the best technical paper at Belgrade University. During my bachelor's, I continued my interest in computational physics. I did a summer internship at Helmholtz-Zentrum Dresden Rossendorf in a group of Dr. Michael Bussmann, where I worked on an improved version of an integration algorithm for their PIC code. 
\par
I continued my master's degree in Belgrade. Still, I did most of my thesis work at the Institute of Physics under the supervision of Milica Milovanović. The title of my master's thesis is: ``Lattice-like structures in Lowest Landau Level'' where we were trying to construct an effective Hamiltonian for a bosonic system on a square lattice that will support fractional quantum Hall effect (FQHE) at $\nu=\frac{1}{2}$. I started my Ph.D. in November 2018, in the Quantum Matter Theory Group at the Lorentz Institute of Leiden University, under the supervision of Prof. Dr. J. Zaanen and Prof. Dr. K.E. Schalm. During my Ph.D., I taught a couple of courses as a teaching assistant, ``Classical Electrodynamics'' and ``Theory of General Relativity''. Alongside the research, I have attended several schools during my Ph.D., DRSTP Schools in High Energy and Condensed Matter Physics, in Brazil and the Netherlands. I have also presented my work at several Physics@Veldhoven conferences.

As of January 1st, 2023, I have started working as a Quantitative Developer for an energy trading company Northpool B.V.


\newpage
\thispagestyle{empty}