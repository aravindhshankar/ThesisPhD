\documentclass[pdftex,a5paper]{dissertation}
\pdfpagewidth=\paperwidth
\pdfpageheight=\paperheight


%% Turn off page numbering for the propositions and make the margins on both
%% sides equal and symmetrical.
\geometry{twoside=false}
\pagestyle{empty}
\geometry{a5paper,margin=10mm}



\begin{document}



%% Specify the title and author of the thesis. This information will be used on
%% both the English and Dutch side and in the metadata of the final PDF.
\title{Numerical Exploration of Statistical Physics \\}
\author{Aleksandar}{Bukva}

\begin{center}

{\Large\titlefont\bfseries Stellingen\\}

\vspace{2mm}
Behorende bij het proefschrift

\medskip



\bigskip

%% Print the title.
{\makeatletter
\titlestyle\bfseries\large\@title
\makeatother}

%% Print the optional subtitle.
{\makeatletter
\ifx\@subtitle\undefined\else
    \titlefont\titleshape\@subtitle
\fi
\makeatother}

%\bigskip
%
%by
%
%\bigskip

% Print the full name of the author.
%\makeatletter
%{\large\titlefont\bfseries\@firstname\ {\titleshape\@lastname}}
%\makeatother

\end{center}

\smallskip

{\small
\begin{enumerate}[leftmargin=*]
\setlength\itemsep{0.8em}

\item Integrable theories can have operators that thermalize.
\vspace{-10pt}
\begin{flushright}
    [\emph{Chapter 2}]
\end{flushright}

\item Multiple matter fields gauged with the same $Z_2$ field exhibit ``registry'' order parameter in the Higgs phase.  
\vspace{-10pt}
\begin{flushright}
    [\emph{Chapter 3}]
\end{flushright}


\item Entanglement entropy of the lattice gauge theories with matter fields at criticality does not grow with the number of additional matter fields. 
\vspace{-10pt}
\begin{flushright}
    [\emph{Chapter 4}]
\end{flushright}

\item Initialization along the edge of chaos is a necessary but not sufficient condition for optimal trainability. 
\vspace{-10pt}
\begin{flushright}
	[\emph{Chapter 5}]
\end{flushright}

\item Generative models can be used to improve calculation of entanglement entropy in lattice gauge field models.
\vspace{-2pt}
\begin{flushright}
	M. Medvidovic, J. Carrasquilla, L. E. Hayward, B. Kulchytskyy, \emph{Generative models for sampling of lattice field theories
	} arXiv:cond-mat/2012.01442
\end{flushright}

\item The recent development of machine learning techniques can significantly improve our current optimization algorithms. But caution must be applied and methods understood rather than mindlessly used.

\item Advancements in computational physics would be vastly accelerated by making codes of research papers publicly available. 

\item The accuracy of the ground-state energy is not sufficient evidence of the proper representation of a ground state in gapless systems.

\item Considering machine learning beyond being merely a profoundly intricate minimization challenge is fruitless. 

\end{enumerate}

\vspace{10pt}

\begin{flushright}

Aleksandar Bukva\\
Leiden, 10$^{\text{th}}$ October 2023

\end{flushright}
}

\end{document}
