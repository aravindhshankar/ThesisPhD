\documentclass[pdftex,a5paper]{dissertation}
\pdfpagewidth=\paperwidth
\pdfpageheight=\paperheight


%% Turn off page numbering for the propositions and make the margins on both
%% sides equal and symmetrical.
\geometry{twoside=false}
\pagestyle{empty}
\geometry{a5paper,margin=10mm}



\begin{document}



%% Specify the title and author of the thesis. This information will be used on
%% both the English and Dutch side and in the metadata of the final PDF.
\title{Strongly correlated electrons in Sachdev-Ye-Kitaev models and Twisted bilayer graphene\\}
\author{Aravindh Swaminathan}{Shankar}

\begin{center}

{\Large\titlefont\bfseries Stellingen\\}

\vspace{2mm}
Behorende bij het proefschrift

\medskip



\bigskip

%% Print the title.
{\makeatletter
\titlestyle\bfseries\large\@title
\makeatother}

%% Print the optional subtitle.
{\makeatletter
\ifx\@subtitle\undefined\else
    \titlefont\titleshape\@subtitle
\fi
\makeatother}

%\bigskip
%
%by
%
%\bigskip

% Print the full name of the author.
%\makeatletter
%{\large\titlefont\bfseries\@firstname\ {\titleshape\@lastname}}
%\makeatother

\end{center}

\smallskip

{\small
\begin{enumerate}[leftmargin=*]
\setlength\itemsep{0.8em}

\item Many crucial aspects of the SYK model persist upon reducing the connectivity of the interactions from all-to-all.
\vspace{-10pt}
\begin{flushright}
    [\emph{Chapter 2}]
\end{flushright}

\item Coupled Yukawa SYK models allow one to compute the Josephson current between two superconductors non-perturbatively in the coupling interaction.  
\vspace{-10pt}
\begin{flushright}
    [\emph{Chapter 3}]
\end{flushright}

\item The wormhole solutions obtained in the Yukawa SYK model only allow transmission of electronic excitations, but not Cooper pairs. 
\vspace{-10pt}
\begin{flushright}
    [\emph{Chapter 3}]
\end{flushright}

\item The Kondo effect can be used as a probe to identify crucial qualitative features of the band structure of twisted bilayer graphene. 
\vspace{-10pt}
\begin{flushright}
    [\emph{Chapter 4}]
\end{flushright}


\item Although the mechanism of superconductivity in the Yukawa SYK model is \textbf{not} of the conventional electron-phonon type, it still shows an isotope effect. Conversely, an observation of the isotope effect is not an immediate guarantee of a BCS like superconducting mechanism. 
\vspace{-2pt}
\begin{flushright}
	I. Esterlis, J. Schmalian \emph{Phys. Rev. B 100, 115132 (2019)} 
\end{flushright}

\item A Moire pattern in the real space lattice of a material is a strong indicator of the possible existence of flat bands in the said material. 

\item The existence of a true quantum critical point has not been confirmed beyond reasonable doubt. This is because any function can be made to look like a power law on a log-log scale for less than a decade.  

\item Arrays of SYK dots in the weak inter-dot tunneling limit can be used to model the phase diagram near the Mott phase of cuprates, but such a description fails in the strange metallic phase.
\vspace{-2pt}
\begin{flushright}
    A. A. Patel, H. Guo, I. Esterlis, S. Sachdev \emph{Science 381,790-793 (2023)} 
\end{flushright}


\item Although aggressively marketed as such, it is unclear whether quantum computers will be useful for understanding much about quantum many body problems. 

\end{enumerate}

\vspace{10pt}

\begin{flushright}

Aravindh Swaminathan Shankar\\
Leiden, 7$^{\text{th}}$ January 2025

\end{flushright}
}

\end{document}
